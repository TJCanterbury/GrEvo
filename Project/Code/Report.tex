\documentclass[fontsize=11pt]{scrartcl}\usepackage[]{graphicx}\usepackage[]{color}

\usepackage[english]{babel}
\linespread{1.5}
\usepackage{array} %table
\usepackage{pgfgantt} %gantt

\newcommand{\HRule}[1]{\rule{\linewidth}{#1}} 	% Horizontal rule

\makeatletter							% Title
\def\printtitle{%						
    {\centering \@title\par}}
\makeatother									

\makeatletter							% Author
\def\printauthor{%					
    {\centering \large \@author}}				
\makeatother							

% --------------------------------------------------------------------
% Metadata (Change this)
% --------------------------------------------------------------------
\title{	
			\HRule{0.5pt} \\						% Upper rule
			\LARGE \textbf{\uppercase{A Network Alignment Based Method for Phylogenetic Analysis of Interdependent Morphological Characters}}	% Title
			\HRule{2pt} \\ [0.5cm]		% Lower rule + 0.5cm spacing
			\normalsize \today			% Todays date
		}

\author{
		Tristan J. Canterbury\\	
    Supervisor: Dr. Martin D. Brazeau\\
		Imperial College London\\	
		MSc Computational Methods in Ecology and Evolution\\
        \texttt{tjc19@ic.ac.uk} \\
}
\IfFileExists{upquote.sty}{\usepackage{upquote}}{}
\begin{document}

    \thispagestyle{empty}		% Remove page numbering on this page

  \printtitle					% Print the title data as defined above
      \vfill
  \printauthor				% Print the author data as defined above
  \newpage
  % ------------------------------------------------------------------------------
  % Begin document
  % ------------------------------------------------------------------------------
  \setcounter{page}{1}

  \section{Keywords}
  \begin{enumerate}
    \item Phylogenetics
    \item Morphology
    \item Software development
    \item Hierarchical characters
    \item Dynamic homology
    \item Comparing networks
  \end{enumerate}

  \section{Introduction} 
  Phylogenetic inference based on morphology allows us to corroborate proposed phylogenies of extant lineages with evidence from the fossil record for which we only have morphological evidence. 
  As most species that have ever lived are extinct, few of which we have fossils for, the value of robust methods for inferring phylogenetic relationships from morphology becomes apparent.
  In particular, problems that do rely heavily on morphological evidence, such as the long unresolved Arthropod head problem \cite{Budd2002,Rempel1975}, might benefit from fresh methods of phylogenetic inference. 
  An issue that arises when we attempt to measure phylogenetic distances based on morphological similarities, from which homology is somewhat arbitrarily inferred, is that many morphological 
  characters do not evolve independently and tend to form hierarchical structures \cite{Hopkins2021}. Taking a wide glance at biology we should expect morphological characters to be interdependent 
  for many different reasons: All morphological characters grow from 
  the same embryo, they are physically connected, most are polygenetic, most alleles that code for a particular character will also code 
  for another (pleiotropy), and they all co-evolve and they may serve similar adaptive purposes and so may converge on similar features. Whilst many of these 
  relationships are hard to ascertain, the physical relationships between these characters are often obvious but underutilised in current quantitative methods
  of phylogenetic analysis as there has been no good way to codify them. \\
  The aim of my project is to represent morphological interdependencies as a network/graph of characters, an anatomical network, and infer phylogenetic distance based on evidence of transformation events between the 
  topologies of these networks, with the assumption that fewer graph transformations means greater parsimony \cite{Grant2004}. Further questions to 
  explore would be to ask what other useful information can be gained from anatomical networks besides phylogenetic distance, how effective is this method in 
  comparison to other methods and how can the software be further optimised. 
  
  \section{Proposed Methods}
  With my goal being to reconstruct ancestral states without making prior assumptions of homology between particular characters, this method of comparing networks must work 
  where there is an unknown node-correspondence (UNC). These node correspondences will instead be estimated using a alignment-based network comparison method. 
  A measure of distance between networks is also needed so that I can infer the most likely phylogenetic relationships between nodes of the tree. 
  Such methods generally belong to the GRAAL family as these algorithms estimate node correspondences based on the topology of the network and outputs a 
  measure of distance between the trees \cite{Tantardini2019}.
  Of the GRAAL family the Lagrangian graphlet-based network aligner (L-GRAAL) is currently the best candidate as it also allows for use of biological information and directional graphs \cite{Malod-Dognin2015}. 
  The main draw back of these methods is that their computational efficiency scales quadratically or worse with the number of nodes. I may have to construct my own prototype of this software in python
  for morphological data however as the C++ source code for the L-GRAAL programme is not available for me to adapt for these purposes.
 
  \section{Anticipated Outputs and Outcomes}
  \begin{itemize}
    \item Development of a method for measuring phylogenetic distances between anatomical networks.
    \item A prototype software package in python and C++ for implementing this new method for tree evaluation.
    \item An analysis of the efficacy of this method for solving the problem of morphological character interdependencies through the example of the Arthropod head problem.
  \end{itemize}

  \section{Project Feasibility}

  \newgantttimeslotformat{stardate}{%
    \def\decomposestardate##1.##2\relax{%
      \def\stardateyear{##1}\def\stardateday{##2}%
    }%
    \decomposestardate#1\relax%
    \pgfcalendardatetojulian{\stardateyear-04-01}{#2}%
    \advance#2 by-1\relax%
    \advance#2 by\stardateday\relax%
  }
    
  \begin{ganttchart}[
    time slot format=stardate, 
    bar/.append style={orange},
    expand chart=\textwidth
    ]{2021.1}{2021.152}
    \gantttitlecalendar{month=name} \\
    \ganttbar{Literature Review}{2021.7}{2021.37} \\
    \ganttlinkedbar{Write Introduction}{2021.10}{2021.61} \\
    \ganttlinkedbar{Write Methods}{2021.30}{2021.90} \\
    \ganttlinkedbar{Data Structure}{2021.30}{2021.37} \\
    \ganttlinkedbar{Develop Anatomical Network Comparison Method}{2021.38}{2021.70} \\
    \ganttlinkedmilestone{Achieved Primary Goal}{2021.70} \\
    \ganttlinkedbar{Write Tree Function}{2021.70}{2021.90} \\
    \ganttlinkedbar{Test on Arthropod Head Problem}{2021.90}{2021.110} \\
    \ganttlinkedbar{Write Results}{2021.70}{2021.110} \\
    \ganttlinkedbar{Write Discussion}{2021.90}{2021.146}  \\
    \ganttlinkedmilestone{Submission 26th August}{2021.146} 
  \end{ganttchart}

  \section{Budget}
  \begin{center}
    \begin{tabular}{ ||p{5cm}|p{2cm}|p{15em}|| } 
     \hline
     Item & Price (£) & Justification \\ 
     \hline
     4 TB External Hard Drive & 100 & Safe storage of project files. \\ 
     \hline
     Books & 300 & References and courses on set theory, graph theory, Information Theory, complex networks and phylogenetic analysis.  \\ 
     \hline
    \end{tabular}
  \end{center}

  \section{Supervisor Declaration}
  \textbf{I have seen and approved the proposal and the budget.}\\
  Primary Supervisor:  Dr Martin D. Brazeau \\
  Signature:\\
   \\
  Date:  08/04/2021

  \bibliographystyle{apalike}
  \bibliography{Morph.bib}
\end{document}
